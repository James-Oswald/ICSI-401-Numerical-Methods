\documentclass{article}
\usepackage[utf8]{inputenc}
\usepackage[normalem]{ulem}
\usepackage{listings}

\usepackage{hyperref}
\hypersetup{
    colorlinks=true,
    linkcolor=red,
    filecolor=magenta,      
    urlcolor=cyan,
}
\usepackage{graphicx}
\usepackage{amsmath}
\usepackage{amsfonts}
\usepackage{amssymb}
\usepackage{array}
\usepackage{float}
\restylefloat{table}
%fancy headers
\usepackage{fancyhdr}
\pagestyle{fancy}
\fancyhf{}
\lhead{ICSI 401 Final}
\rhead{\thepage}


\author{James Oswald}
\date{December 6th, 2020}
\title{ICSI 401 Final}


\begin{document}
\maketitle
\thispagestyle{fancy}

\begin{enumerate}
    \item[1.] (10 points) Calculate the Lagrange form of the interpolating polynomial for the following points: $(0, 1),(1, 3),(2, 12)$.
            %x1  y1  x2  y2  x3  y3
    \newline
    \newline
    I solve by using the formula for the Lagrange form of the interpolating polynomial with three points and plugging in our respective values.
    \begin{align*}
        P(x) &= \frac{(x-x_2)(x-x_3)}{(x_1-x_2)(x_1-x_3)}y_1 + \frac{(x-x_1)(x-x_3)}{(x_2-x_1)(x_2-x_3)}y_2 + \frac{(x-x_1)(x-x_2)}{(x_3-x_1)(x_3-x_2)}y_3 \\
        P(x) &= \frac{(x-1)(x-12)}{(0-1)(0-1)}1 + \frac{(x-0)(x-2)}{(1-0)(1-2)}3+ \frac{(x-0)(x-1)}{(2-0)(2-1)}12\\
        P(x) &= \frac{(x-1)(x-2)}{(-1)(-2)}1 + \frac{(x)(x-2)}{(1)(1-2)}3+ \frac{(x)(x-1)}{(2)(1)}12\\
        P(x) &= \frac{x^2-3x+2}{2} + \frac{x^2-2x}{-1}3+ \frac{x^2-x}{2}12 \\
        P(x) &= \frac{x^2-3x+2}{2} + \frac{-6x^2+12x}{2}+ \frac{12x^2-12x}{2}\\
        P(x) &= \frac{x^2-3x+2-6x^2+12x+12x^2-12x}{2}\\
        P(x) &= \frac{7x^2-3x+2}{2}\\
        P(x) &= \frac{7}{2}x^2-\frac{3}{2}x+1
    \end{align*}
    Thus the Lagrange form of the interpolating polynomial is $\frac{7}{2}x^2-\frac{3}{2}x+1$
    
    \newpage
    \item[2.] (10 points) Complete the divided differences table for the above points. Your table should be in the following form:\newline\newline
    \begin{tabular}{ c|c|c } 
        $f[0]$ & X & X \\
        \hline
        $f[1]$ & $f[0, 1]$ & X \\
        \hline
        $f[2]$ & $f[1, 2]$ & $f[0, 1, 2]$ \\ 
    \end{tabular}\newline\newline
    Here, an X indicates that there should NOT be a number in that entry.
    \newline
    \newline
    In other words, what is expected is that you compute $f[0], f[1], f[0, 1]$, etc.
    We begin by calculating our first column based off of the points provided.
    \begin{align*}
        f[0] = 1,  f[1] = 3, f[2] = 12
    \end{align*}
    Next we use the interpolation formula to calculate terms in the second column:
    \begin{align*}
        f[a, b] &= \frac{f[b]-f[a]}{b-a}\\
        f[0, 1] &= \frac{f[1]-f[0]}{1-0} = \frac{3-1}{1-0} = 2\\
        f[1, 2] &= \frac{f[2]-f[1]}{2-1} = \frac{12-3}{2-1} = 9\\
    \end{align*}
    Finally we use:
    \begin{align*}
        f[a, b, c] &= \frac{f[b, c]-f[a, b]}{c-a}\\
        f[0, 1, 2] &= \frac{f[1, 2]-f[0, 1]}{2-0} = \frac{9-2}{2} = 3.5\\
    \end{align*}
    Thus our final divided differences table is:
      \begin{center}
        \begin{tabular}{ c|c|c } 
             1 &  &  \\ 
             \hline
             3 & 2 &  \\
             \hline
             12 & 9 & 3.5\\ 
        \end{tabular}
    \end{center}
    
    \newpage
    \item[3.](10 points) Consider the following divided differences table for the points $(0, 1),(1, 2),(2, 4)$.\newline\newline
    \begin{tabular}{ c|c|c } 
        \hline
        $1$ &  &  \\
        \hline
        $2$ & $1$ &  \\
        \hline
        $4$ & $2$ & $1/2$ \\ 
    \end{tabular}\newline\newline
    Use the divided differences table to write down the Newton form of the interpolating polynomial for the given points.
    \newline
    \newline
    To calculate the Newton form of the interpolating polynomial, we use the formula:
    \[P(x) &= f[x_0]+f[x_0, x_1](x-x_0)+f[x_0, x_1, x_2](x-x_0)(x-x_1)\]
    Thus we can use the diagonal values in the provided by the divided differences table to plug in:
    \begin{align*}
        P(x) &= 1+1(x-0)+\frac{1}{2}(x-0)(x-1)\\
        P(x) &= 1+x+\frac{1}{2}x^2-\frac{1}{2}x\\
        P(x) &= \frac{1}{2}x^2+\frac{1}{2}x+1\\
    \end{align*}
    Thus the Newton form of the interpolating polynomial for these points is $\frac{1}{2}x^2+\frac{1}{2}x+1$
    
    \newpage
    \item[4.] (10 points) Suppose that we wish to find a piecewise polynomial interpolation for the points $(x_0, y_0) = (0, 1),(x_1, y_1) = (1, 3),(x_2, y_2) = (2, 1)$. In particular, we want two polynomials $P_0(x)$ and $P_1(x)$ satisfying various matching conditions, and we define
    \[P(x) = \begin{cases} 
        P_0(x) & x\in[x_0,x_1) \\
        P_1(x) & x\in[x_1,x_2] \\ 
    \end{cases}\]
    Write down a condition on the values of $P_0$ and $P_1$ that is necessary and sufficient to conclude that $P(x)$ is continuous on the interval $[x0, x2]$.\\
    \newline
    \newline
    To prove that $P(x)$ is continuous on the interval $[x0, x2]$ (assuming that it is a given that $P_0(x)$ and $P_1(0)$ are continuous) it is necessary and sufficient to prove that $P_0(x_1) = P_1(x_1)$. If this condition is met then $P(x)$ is guaranteed to be continuous on the interval $[x0, x2]$
    
    \newpage
    \item[5.] (10 points) The following pseudocode implements Horner’s rule for polynomial evaluation
    \begin{verbatim}
% Implements Horner’s method for evaluating a polynomial at a point x.
%
% Inputs:
% * coefficients: a row vector of polynomial coefficients, with the lowest-degree
% coefficient first. For example, [1, 2, 3] corresponds to the polynomial
% p(x) = 1 + 2x + 3x^2.
% * x: a floating point number.
%
% Output:
% The value of p(x).
%
function y = horner(coefficients, x)
    degree = size(coefficients)-1;
    degree = degree(2);
    if (degree == 0)
        y = coefficients(1)
    else
        % degree > 1.
        z = horner(coefficients(2:(degree+1)), x)
        y = coefficients(1) + x * z;
    end
end
    \end{verbatim}
    \begin{tabular}{ c|c|c } 
        coefficients & x & y \\ 
        \hline
        [1, 1, 3, 1] &  &  \\
        \hline
        [1, 3, 1] &  & \\ 
        \hline
        [3, 1] & 1 & 3 \\
        \hline
        [1] & N/A & 1
    \end{tabular}\newline
    Complete either table using the corresponding pseudocode for full credit.
    \newline
    \newline
    I use the pseudocode to calculate the correct values for the table: \newline\newline
    \begin{tabular}{ c|c|c } 
        coefficients & x & y \\ 
        \hline
        [1, 1, 3, 1] &  &  \\
        \hline
        [1, 3, 1] &  & \\ 
        \hline
        [3, 1] & 1 & 3 \\
        \hline
        [1] & N/A & 1
    \end{tabular}\newline
\end{enumerate}


\end{document}